\chapter{吐(qian)槽(yan)}
这是我在离高考还有三个月的时候作死写出来的一些东西,本来是为了给同学讲故事而准备的,对以后几届数学、物理竞赛和自主招生的同学应该也有用,现在发布在网上。

我希望把数学和物理书上晦涩难懂的内容用人话重新讲一遍,让同学们看了之后就能拿去干活。这些东西主要分为数学基础,物理基础,高考范围内的一些黑科技,以及高考更加不可能考的东西。很多地方没有严谨的数学证明,或者是强行猜出来的,欢迎学数学的小伙伴来打我。

这些东西的难度有很大跨度,你可以挑自己看得懂的部分看,看不懂也不用背下来。正常情况下每章大概$10$到$30$分钟看完,学(sang)有(xin)余(bing)力(kuang)的同学还可以参考其他书。\emph{这样的字体}(以及括号里的内容)表示看起来很厉害,但是不用管它的东西。

这些东西不能代替课本和教辅书,只能作为补充,特别是练习题要从其他地方找,有些东西刷了足够多的题才能理解。文中的练习不多,建议把每个式子看完之后自己推导一遍。

因为篇幅原因,这里没有积分表、公式表、常数表之类的东西,你可以去查其他书和网站。

文章使用XeLaTeX排版,有些插图会乱跑,而且会出现很多空白页和页边,可以拿去打草稿或者召唤英灵费马(你应该知道怎么在电子书上面打草稿)。部分插图使用Mathematica绘制。

目前我不打算出版纸质书。如果要打印建议双面打印。转载请注明版权,请勿用于商业用途。

文章会不定期更新,之前发布的内容可能会随时修改。($\leftarrow$你信吗)

Github主页:\url{https://github.com/woct0rdho/project-asteria}

备用地址:\url{https://mega.nz/#F!dIM2AY5D!u2FKDMgruY9l6l8znshkfw}

如果发现错误或者有建议,欢迎与我联系:\url{mailto://woctordho@outlook.com}

啊。。好像跟周杰伦的专辑撞上了
