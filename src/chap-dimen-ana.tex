\chapter{量纲分析与原子物理}
\section{量纲}
量纲是一种和单位差不多的东西,每个物理量有它的单位,也有它的量纲。我们用方括号大写字母表示量纲(虽然打草稿的时候肯定不会这么写),常用的量纲有长度$[L]$,质量$[M]$,时间$[T]$,电流$[I]$和温度$[\Theta]$。($\Theta$的读音是theta,而拉丁文的“温度”是thermo)

比如长度$l$这个物理量的量纲是$[L]$。不管它的单位是米、厘米还是英寸,它的量纲都是$[L]$。

再比如公式$F=m a$,$m$的量纲是$[M]$,$a$的量纲是$[L] [T]^{-2}$,所以$F$的量纲是$[L] [M] [T]^{-2}$。一个等式两边的量纲必须相等,做完题目之后记得检查这一点。同一个矩阵的元素量纲必须相同,否则矩阵乘法就没有意义。

下面是一些常见物理量的量纲:

\begin{tabular}{lrlr}
能量$E$与力矩$M$ & $[L]^2 [M] [T]^{-2}$ & 角动量$L$ & $[L]^2 [M] [T]^{-1}$ \\
引力常量$G$ & $[L]^3 [M] [T]^{-2}$ & 静电力常量$k$ & $[L]^3 [M] [T]^{-4} [I]^{-2}$ \\
电压$U$ & $[L]^2 [M] [T]^{-3} [I]^{-1}$ & 电阻$R$ & $[L]^2 [M] [T]^{-3} [I]^{-2}$ \\
电场强度$E$ & $[L] [M] [T]^{-3} [I]^{-1}$ & 磁感应强度$B$ & $[M] [T]^{-2} [I]^{-1}$ \\
真空介电常量$\epsilon_0$ & $[L]^{-3} [M]^{-1} [T]^4 [I]^2$ & 真空磁导率$\mu_0$ & $[L] [M] [T]^{-2} [I]^{-2}$ \\
\end{tabular}

【练习】用你熟悉的公式把这些东西推导一遍。现在不用背下来,用得多了就背下来了。

如果你现在不能理解量纲是什么,理解成单位也没问题。
\section{量纲分析}
考虑这样一个问题:在太空中放一个水球,然后戳它一下,由于表面张力作用,它会发生振动,求振动频率。

首先猜一猜频率可能与哪些物理量有关,这一步需要传说中的“物理直觉”。我们先猜频率$f$与球的半径$r$、水的密度$\rho$、水的表面张力系数$\sigma$有关。

(介绍一下表面张力系数:液体与固体或者气体的分界面会产生一个势能$E=\sigma S$,叫作表面能。$S$是分界面的面积,液体会尽量减小$S$来使能量最低。)

$f$的量纲是$[T]^{-1}$,$r$的量纲是$[L]$,$\rho$的量纲是$[L]^{-3} [M]$,$\sigma$的量纲是$[M] [T]^{-2}$。

设$f=A r^a \rho^b \sigma^c$,$A$是一个无量纲的系数。等式两边的量纲必须相等,因此对$[L],[M],[T]$分别列出方程:
\begin{equation*}
\begin{cases}
a-3b=0 \\
b+c=0 \\
-2c=-1
\end{cases}
\end{equation*}

解得$a=-\frac{3}{2},b=-\frac{1}{2},c=\frac{1}{2}$,也就是说$f=A \sqrt{\frac{\sigma}{\rho r^3}}$。量纲分析并不能确定$A$,但是有些问题中$A$并不重要,只要知道$r,\rho,\sigma$变化时$f$大致的变化情况就行了。(想想之前讲的复杂度理论)事实上,用一些流体力学方法可以算出$A$的最小值为$2\sqrt{2}$。

我们要相信物理学中的每个系数都是有它出现的原因的,比如$2$通常是$x^2$求导留下的,$\pi$通常在画了一个圆或者球之后出现,至于$\sqrt{57 \pi}$这样奇怪的系数就没有道理出现在基本的公式里。

如果单位制合适,$A$的大小不会跟$1$相差太多。比如你用毫米量一根杆子的长度是$1000$多,只要改用米来量就和$1$差不多了。

现在回过头一看:$\rho r^3$和质量$m$的量纲是相同的。如果一开始猜$f$与$m,\sigma$有关,那么这道题就秒杀了。量纲分析能否成功,以及过程有多复杂会受一开始猜的参量影响,如果参量猜得太多,方程数量就不够;如果参量猜得太少,就凑不出需要的量纲。

再来看一个复(zhuang)杂(bi)的例子:把两块无限大的薄导体板放在真空中,根据量子电动力学,两块板之间会出现一些电磁场的涨落,所以它们会相互吸引,这个效应称为\emph{卡西米尔效应}。我们来求吸引力造成的压强。

卧槽这是什么鬼!但是我们要相信,给我们做的题目总是有办法做出来的。因为板是无限大的,所以它们的质量和面积都没有意义。(两块板之间的吸引力大小也没有意义,但是单位面积的吸引力、也就是压强是有意义的)而且板是薄的,也就是说它们的厚度没有意义。但是两块板的间距$d$可以当作一个参量。

为了凑出压强的量纲,还要用一些物理常量:这道题与量子有关,所以可能需要普朗克常量$h$;还与电磁场有关,所以可能需要光速$c$和真空介电常量$\epsilon_0$。

(做这样的题目需要一些物理常识。如果题目与相对论有关,可能需要光速$c$;如果题目与万有引力或者广义相对论有关,可能需要引力常量$G$;如果题目与分子运动有关,可能需要玻尔兹曼常量$k$。这里的吸引力并不是万有引力)

设压强$p=A d^\alpha h^\beta c^\gamma \epsilon_0^\delta$。($c$和$d$都被用掉了好悲伤)

对$[L],[M],[T],[I]$分别列出方程:($p$的量纲是$[L]^{-1} [M] [T]^{-2}$,$h$的量纲是$[L]^2 [M] [T]^{-1}$)
\begin{equation*}
\begin{cases}
\alpha+2\beta+\gamma-3\delta=-1 \\
\beta-\delta=1 \\
-\beta-\gamma+4\delta=-2 \\
\delta=0
\end{cases}
\end{equation*}

诶我写到这里才想起来$\delta=0$,所以其实我们并不需要$\epsilon_0$。解得$\alpha=-4,\beta=1,\gamma=1$,所以$p=A \frac{h c}{d^4}$。
\section{无量纲数}
只有量纲相同的东西才能比较大小,比如你可以说一根棍子比另一根棍子长,但是不能说一根棍子的长度比另一根棍子的重量大。

事实上,比较物理量大小的本质是比较无量纲数的大小,比如一根棍子的长度$l_1$比另一根棍子的长度$l_2$大,就是说无量纲数$\frac{l_1}{l_2}>1$。不管用米、厘米还是英寸来量它们的长度,都是第一根棍子比较长。无量纲数可以描述一个物理系统的性质,而与单位制无关。再比如阻尼振动是欠阻尼、临界阻尼还是过阻尼,就是由无量纲数$\frac{\omega_0}{\beta}$决定的。

$\sin x,\exp x$等函数的自变量必须是无量纲数,比如$\sin \omega t$,$\exp -\frac{\hbar \omega}{k T}$。角度也是无量纲的,比如角速度$\omega$的单位是$\unit{rad/s}$,量纲是$[T]^{-1}$。

在高考中我们经常遇到$t=\sqrt{\frac{2 h}{g}}$,$v=\sqrt{2 g h}$等等,在打草稿的时候可以设$t_0=\sqrt{\frac{h}{g}}$,$v_0=\sqrt{g h}$,那么$t=\sqrt{2} t_0$,$v=\sqrt{2} v_0$。计算的时候只要算前面的系数就行了,后面的字母可以根据量纲补上。比如$x=v t$,那么$x$的系数是$2$,后面是$h$。$a=\frac{v}{t}$,那么$a$的系数是$1$,后面是$g$。更复杂的例子等要用的时候再讲。
\section{精细结构常数;秒杀原子物理题}
现在来看看玻尔的氢原子模型。假设原子核与电子之间的力满足库仑定律$F=\frac{1}{4 \pi \epsilon_0} \frac{e^2}{r^2}$,电子的运动满足牛顿第二定律$F=m_e \frac{v^2}{r}$,以及量子化条件$m_e v r =n \hbar (n=1,2,3,\dots)$。

($\hbar=\frac{h}{2 \pi}=1.05 \times 10^{-34} \unit{J \cdot s}$,叫作约化普朗克常量,读作h bar。在原子物理中,频率$f$和角速度$\omega$相差一个系数$2 \pi$,所以有时候用$\hbar$比$h$更方便)

解得电子轨道半径$r=\frac{4 \pi \epsilon_0 \hbar^2 n^2}{e^2 m_e}$,速度$v=\frac{e^2}{4 \pi \epsilon_0 \hbar n}$,还可以计算出能量$E=\frac{1}{2} m_e v^2-\frac{1}{4 \pi \epsilon_0} \frac{e^2}{r}=-\frac{e^4 m_e}{2 (4 \pi \epsilon_0)^2 \hbar^2 n^2}$,代入数值可以算出我们熟悉的$E=-\frac{13.6}{n^2} \unit{eV}$。

(这些物理量的数值在卡西欧计算器上都有,但是卡西欧直接算$e^4 m_e$会因为数值太小而报错,需要拆成两部分来算)

【练习】用能量最低原理和量子化条件把这些东西推导一遍。

有没有简便一点的计算方法呢?我们定义一个无量纲数:精细结构常数$\alpha=\frac{e^2}{4 \pi \epsilon_0 \hbar c} \approx \frac{1}{137}$。这是用$e,\epsilon_0,\hbar,c$能凑出的最简单(指数是整数且最小)的无量纲数。注意我们并没有选电子质量$m_e$,因为它是电子的特性;而$e$不仅是电子电量,而是电荷的基本单位,它的地位比$m_e$更基本。

$\alpha$里面有一个系数$\frac{1}{4 \pi}$,因为我们经常需要在一个电荷周围画一个球面然后算它的面积,所以为了方便就把$\frac{1}{4 \pi}$放进去了。

现在我们发现$v=\frac{\alpha}{n} c$,$r=(\frac{\alpha}{n})^{-1} \frac{\hbar}{m_e c}$,$E=-\frac{1}{2} m_e v^2=-\frac{1}{2} (\frac{\alpha}{n})^2 m_e c^2$。这样一来这些结果的物理意义就比较清楚,而且比较容易背下来。

$m_e v r =n \hbar$里$n$和$\hbar$是相乘的,所以$\alpha$肯定会与$n$相除。$\frac{\hbar}{m_e c}$是$\hbar,m_e,c$能凑出的最简单的无量纲数,它叫作电子的约化康普顿波长,讲相对论的时候说不定还会碰到它。

【练习】如果原子核的电量为$+Z e$,核外只有一个电子,求电子的轨道半径和能量,想一想要对$\alpha$作怎样的修改。
\section{数量级估计}
在热学中我们知道单原子气体分子热运动的平均动能$E_k=\frac{1}{2} m v^2=\frac{3}{2} k T$,$v=\sqrt{\frac{3 k T}{m}}=\sqrt{\frac{3 R T}{\mu}}$。如果把系数当成$1$,我们可以认为$v_T=\sqrt{\frac{R T}{\mu}}$就是分子热运动大致的速度,多原子气体也差不了多少。温度越高,分子质量越小,热运动就越剧烈。空气的$\mu=29 \unit{g/mol}$,我们认为常温是$T=300 \unit{K}$,那么$v_T=3 \times 10^2 \unit{m/s}$。氢气则有$v_T=1 \times 10^3 \unit{m/s}$。(这种估算一般只需要一位有效数字,甚至只需要数量级)

在量子力学中我们又知道德布罗意波长$\lambda=\frac{h}{m v}=\frac{h N_A}{\mu v}$。如果$v=v_T$,我们把这个波长叫作分子的热波长$\lambda_T$,也就是热运动引起的位置不确定程度。空气在常温下$\lambda_T=5 \times 10^{-11} \unit{m}$。

而在经典的热学中,空气满足$p V=n R T$,设两个空气分子的距离为$d$,那么$V=n N_A d^3$,所以$d=(\frac{R T}{N_A p})^{\frac{1}{3}}=3 \times 10^{-9} \unit{m}$。$d \gg \lambda_T$,所以分子之间的量子效应是不显著的。如果温度降低,$\lambda_T$增大,$d$减小,$\lambda_T$接近甚至超过$d$,量子效应就会很显著,这就是传说中的凝聚态物理学研究的东西。

$k T$具有能量的量纲,估算分子的能量时,可以认为温度为$T$时能量就是$k T$。

比如我们要估算核聚变的温度,可以认为核聚变就是把两个质子从无穷远处移到碰到一起为止,这个过程中电势能升高,忽略其他相互作用。但是两个质子之间的距离如果为$0$,电势能就会变得无穷大,什么叫碰到一起呢?可以认为它们的距离是原子核的大小$d=10^{-15} \unit{m}$。现在可以列出$\frac{e^2}{4 \pi \epsilon_0 d}=k T$,所以$T=\frac{e^2}{4 \pi \epsilon_0 d k}=10^{10} \unit{K}$。($k T$是一个还是两个质子的动能就不管了,数量级没有区别)

为了做这样的题目,平常可以留意一些东西的大致数值,比如水的折射率是1.33(稀溶液不会差太多),$1 \unit{g}$ TNT爆炸的能量是$4 \times 10^3 \unit{J}$,地磁场的强度是$5 \times 10^{-5} \unit{T}$,铜的电阻率是$2 \times 10^{-8} \unit{\Omega \cdot m}$,樱花瓣飘落的速度是每秒$5$厘米等等,说不定哪天可以用上。(建议背一下地球、月球、太阳、火星的半径、轨道半径、质量、温度)
\section{国际单位制}
(这些内容可以作为常识了解一下)

国际单位制中有七个基本物理量:长度、质量、时间、电流、温度、发光强度、物质的量。它们的基本单位分别是米、千克、秒、安培、开尔文、坎德拉、摩尔。

很久以前测量电流比电量更方便,所以基本物理量是电流而不是电量。发光强度表示特定方向上眼睛感受到的光的强度,它与光携带的能量并不成正比,而要通过复杂的关系来换算,它的量纲也比较特殊。物质的量专门有一个量纲,但是不太常用。

接下来讲这些基本单位的定义:

铯-133原子发出的某种电磁波的周期的$9192631770$倍为$1$秒。(当然这些数字不用记得很仔细)

光在真空中、在$1/299792458$秒内走过的长度为$1$米。

国际千克原器的质量为$1$千克。(国际千克原器是保存在巴黎的一个金属块,是不是很扯蛋)

真空中有两根无限长、无限细的平行直导线,相距$1$米,往里面通等大的电流,当它们受到每米$1$牛顿的安培力时,通的电流就是$1$安培。

开尔文现在的定义需要一些热力学知识,这里先不讲了。温度有一个特殊之处:其他单位只需要定义相对值,而不需要定义绝对值,比如$1$米的长度需要我们定义,而$0$米并不需要我们定义。但是物理学当中存在一个最低温度——绝对零度,我们需要定义$0 \unit{K}$等于这个温度。以前的一段时间里是这样定义的:一个标准大气压下水的凝固点为273.15K,沸点为373.15K,这样既可以确定$0 \unit{K}$这一温度,又可以确定$1 \unit{K}$这一温度差。

0.012千克碳-12的原子数为1摩尔。

坎德拉的定义比较复杂,这里也不讲了。

近年来许多物理学家希望对上面这些定义作一些修改。特别是千克的定义,很大程度上是人为规定的。而安培的定义在实际操作中十分困难。

米的定义就比较好:先定义秒,再定义光速$c=299792458 \unit{m/s}$,就有了米的定义。物理学家希望用这种方式定义其他单位,比如定义普朗克常量$h=6.63 \times 10^{-34} \unit{m^2 \cdot kg \cdot s^{-1}}$,就有了千克的定义。当然三位有效数字是远远不够的,需要用精确的实验手段把$h$测到十几位有效数字,这样的定义才能被大家同意。类似地可以用元电荷$e$定义安培,用玻尔兹曼常量$k$定义开尔文。新的国际单位制有望在2018年被一帮开会的人通过。

顺便说一下,$c$是定义出来的精确值,$\mu_0=4 \pi \times 10^{-7} \unit{m \cdot kg \cdot s^{-2} \cdot A^{-2}}$也是,所以$\epsilon_0$是可以算出来的精确值。
